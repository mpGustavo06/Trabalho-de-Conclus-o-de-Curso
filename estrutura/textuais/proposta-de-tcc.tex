% PROPOSTA DE TRABALHO DE CONCLUSÃO DE CURSO-----------------------------------------------------------

\chapter{PROPOSTA DE TRABALHO DE CONCLUSÃO DE CURSO}
\label{chap:proposta}

\section{TÍTULO}
\label{sec:titulo}
% Informe o título do trabalho-------------------------------------------------------------------------
Sistema de Notificação Automatizada para Criadores de Conteúdo com Integração Multicanal.
%------------------------------------------------------------------------------------------------------

\section{MODALIDADE DO TRABALHO}
\label{sec:modalidade}
% Indique a Modalidade do Trabalho---------------------------------------------------------------------
% Opções:
% - Pesquisa
% - Desenvolvimento de Sistemas
Trabalho Tecnológico.
%------------------------------------------------------------------------------------------------------

\section{ÁREA DO TRABALHO}
\label{sec:area}
% Indique a Área do Trabalho---------------------------------------------------------------------------
Sistemas distribuídos com foco em integração de APIs e comunicação em tempo real.
%------------------------------------------------------------------------------------------------------

\section{RESUMO}
\label{sec:resumo}
% Resumo do Trabalho-----------------------------------------------------------------------------------
% (máximo de 200 palavras)
% Um resumo deve informar a essência do projeto de maneira resumida, mas completa. Os leitores devem ter uma ideia razoavelmente clara do projeto após ter lido o resumo. Basicamente deve-se colocar informações referentes a finalidade da pesquisa, procedimentos que serão utilizados, observações e dados a serem coletados, resultados esperados (substitua este texto pelo resumo do trabalho).-------------
A comunicação ágil e automatizada entre plataformas digitais tornou-se essencial para criadores de conteúdo e comunidades \textit{online}. Esta proposta de Trabalho de Conclusão de Curso tem como objetivo o desenvolvimento de uma aplicação Web que monitora eventos de transmissão ao vivo e publicação de novos vídeos em serviços como YouTube e Twitch, enviando notificações automáticas para grupos em WhatsApp e Telegram. A solução empregará integração via APIs, autenticação segura e um painel de administração para configuração personalizada de notificações e canais de destino. O projeto visa oferecer uma ferramenta prática e escalável que reduza o esforço manual na divulgação de conteúdos, promovendo maior eficiência na gestão de comunidades virtuais.
%------------------------------------------------------------------------------------------------------

